\chapter{Project Vision}
\label{ch:vision}
\section{Problem Statement}
Medical students often face challenges in gaining hands-on medical experience due to limited availability of actual medical cases, ethical concerns, and high costs associated with traditional medical training methods. This project aims to address these challenges by leveraging the metaverse to create a virtual medical training platform that provides realistic medical scenarios and opportunities to practice in a risk-free environment.

\section{Business Opportunity}
The business opportunity is to offer an innovative, cost-effective and affordable medical training solution to medical students and institutions. A metaverse-based medical training platform can potentially reduce the costs associated with traditional medical training and expand access to high-quality training experiences, thereby improving the skills and readiness of future physicians.

\section{Goals}
The main goals of this project are:
Develop a metaverse-based medical training platform that accurately simulates medical procedures.
To provide medical students with a risk-free environment to practice medical skills.
To enhance the learning experience and improve medical competence among medical students.
Reduce costs and ethical concerns associated with traditional medical training methods.

\section{Project Scope}
The project will involve the development of a metaverse-based medical training environment that will cover a range of medical procedures across different medical specialties. It will include creating realistic medical simulations, integrating haptic feedback devices for tactile training and providing medical students with access to hands-on practice. The project will primarily focus on the technical and educational aspects of metaverse integration.

\section{Limitations}
Constraints include budgetary limitations for technology development and software integration, technology limitations related to metaverse platforms, and the need to adhere to medical training and ethical standards. The project will also need to operate within a set time frame and must comply with relevant regulations.

\section{Description of Stakeholders}
Key stakeholders in this project include medical students, physician educators, healthcare institutions, and potentially patients who may benefit from better trained medical professionals.

\subsection{Stakeholder Summary:}
Medical students are the primary beneficiaries and users of the metaverse-based medical training platform. They can gain valuable hands-on experience that is often limited in traditional training.
Medical educators are stakeholders involved in developing educational content and ensuring compliance with medical curriculum standards.
Healthcare institutions can support or adopt the metaverse training platform to improve the skills and readiness of medical students and residents.
\subsection{Key High Level Objectives and Stakeholder Issues:}
Medical students seek practical medical training to supplement their theoretical knowledge and gain expertise in various medical procedures.
The goal of medical educators is to provide an effective and safe medical training environment that is consistent with the requirements of the medical curriculum.
Healthcare facilities are interested in producing well-trained medical professionals who are able to provide quality care.