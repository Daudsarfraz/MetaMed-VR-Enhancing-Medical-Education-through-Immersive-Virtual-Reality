\chapter{Project Vision}
\label{ch:vision}
\section{Problem Statement}
The project aims to provide medical students with hands-on experience through a virtual medical training platform, leveraging the metaverse to provide realistic scenarios and risk-free practice opportunities, overcoming challenges like limited availability and high costs.

\section{Business Opportunity}
The business opportunity involves offering a cost-effective, metaverse-based medical training platform to medical students and institutions, potentially reducing traditional costs and improving future physician skills.
\section{Goals}
The project aims to create a metaverse-based medical training platform, offering a risk-free environment for medical students to practice skills, enhance learning, and reduce costs and ethical concerns.

\section{Project Scope}
The project aims to create a metaverse-based medical training environment, incorporating realistic simulations, haptic feedback devices, and hands-on practice for various medical specialties.
\section{Limitations}
The project faces constraints such as budget constraints, metaverse platform technology limitations, ethical standards, time constraints, and regulatory compliance.

\section{Description of Stakeholders}
Key stakeholders in this project include medical students, physician educators, healthcare institutions, and potentially patients who may benefit from better trained medical professionals.

\subsection{Stakeholder Summary:}
Medical students benefit from metaverse-based training, gaining hands-on experience. Medical educators develop content, and healthcare institutions can adopt the platform to enhance students' skills and readiness.
\subsection{Key High Level Objectives and Stakeholder Issues:}
Medical students seek practical training to enhance theoretical knowledge, while medical educators aim to create a safe, effective environment for quality care in healthcare facilities.