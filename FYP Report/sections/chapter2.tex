\chapter{Literature overview}
\label{ch:lit-review}
The integration of metaversion, a virtual, interconnected digital realm, into the healthcare field represents a promising frontier in medical technology and patient care. As technology continues to advance, metaverse offers innovative solutions to long-standing challenges in the healthcare industry, from medical training and telemedicine to mental health therapy and patient empowerment. This literature review aims to provide a comprehensive examination of current research, technology and developments in the use of metaversion in healthcare. By examining the historical perspective, technological foundations, practical applications, ethical considerations, and future prospects of metaversion in healthcare, this review seeks to illuminate the current state of knowledge and pave the way for further advances in this rapidly evolving field.
\section{All one needs to know about metaverse: A complete survey on technological singularity, virtual ecosystem, and research agenda \cite{lee2021all}\cite{BookChapter1}}
\textbf{Methodology:}\\The article provides a comprehensive framework for understanding the evolution of the metaverse, exploring state-of-the-art technologies and ecosystems. It identifies core technologies such as augmented reality, artificial intelligence, and blockchain that are driving the transition to the metaverse. The paper discusses user-centric factors like avatars, content creation, and virtual economies crucial for building a sustainable metaverse. It proposes a specific research agenda and offers a blueprint for future studies and innovations in this field. \\
\textbf{Limitations:}\\The article provides a broad overview of enabling technologies and user-centric factors in the metaverse's evolution, rather than conducting an in-depth analysis. The proposed research agenda may need further refinement as technologies and user needs evolve. However, the article lacks extensive discussion on potential challenges and risks, such as ethical issues, inequality, and impacts on physical reality. It also relies primarily on a survey-based approach, without empirical or case study support for its claims and frameworks. \\
\textbf{Results:}\\Comprehensive framework for examining metaverse development.
Concrete research agenda for metaverse development
\section{Healthcare in metaverse: A survey on current metaverse applications in healthcare\cite{bansal2022healthcare}\cite{JournalArticle12}}
\textbf{Methodology:}\\
The article overviews metaverse applications in healthcare, focusing on telemedicine, clinical care, education, mental health, physical fitness, veterinary medicine, and pharmaceuticals. It highlights recent technological developments, potential improvements, and key challenges, including connectivity, privacy, security, integration, interoperability, user experience, and VR/AR technical issues. The need for sustainable and innovative healthcare solutions using metaverse technology is emphasized.\\
\textbf{Limitations:}\\Existing healthcare systems limitations revealed during COVID-19 pandemic. Surge in healthcare innovation using virtual environments for alternative systems. \\
\textbf{Results:}\\It underscores the need to address challenges related to connectivity, privacy, security, integration, interoperability, user experience, and technical issues associated with VR and AR technologies before fully embracing the metaverse for healthcare.
\section{Medical Metaverse: Technology, Applications, Challenges and Future\cite{shao2023medical}}
\textbf{Methodology}\\The paper reviews metaverse technologies and applications in healthcare, gathering information from existing studies to understand the current landscape. It explores scenarios for efficient diagnosis, education, and treatments, and identifies challenges in implementation while proposing solutions for integrating this technology into healthcare settings.\\
\textbf{Limitations:}\\The paper offers a comprehensive review of the Metaverse for healthcare, covering applications, technologies, and challenges. However, it lacks specific case studies or real-world implementations. Limited by available literature, unpublished data or alternative perspectives may be excluded.\\
\textbf{Results:}\\ Review of technologies and applications of the metaverse. 
Exploration of potential and future direction in healthcare.
\section{The Metaverse for Healthcare: A Survey of Potential Applications, Challenges, and Future Directions.\cite{yendurimetaverse}\cite{JournalArticle10}}
\textbf{Methodology:}\\The paper reviews the use of the Metaverse in healthcare, highlighting applications, technologies, and projects. It identifies challenges and proposes future research directions. The paper examines enabling technologies like AI, VR, AR, IoMT, robotics, and quantum computing for personalized patient care and offers insights for overcoming challenges to ensure ongoing advancement.\\
\textbf{Limitations:}\\Challenges in adopting AI-enabled Metaverse in healthcare include the need for high-speed communication, massive computation capabilities, security concerns, data loss, device heterogeneity, and cost implications.  affecting its comprehensiveness.\\
\textbf{Results:}\\
Medical education benefits from VR in learning body structures, enhancing the quality of future doctors. However, AI-enabled Metaverse poses risks to patient privacy and ethical issues, potentially leading to medical errors. Edge computing is crucial for real-time data retrieval in the Metaverse, but it requires additional edge devices, posing network scalability challenges.
\section{Revolutionizing Medical Education with Metaverse\cite{baskar2022revolutionizing}\cite{JournalArticle9}}
\textbf{Methodology:}\\Interdependence of characteristics in virtual teaching model.
Overcoming obstacles through technological advancements in education.\\
\textbf{Limitations:}\\The paper lacks discussion on specific technical challenges, adoption barriers like cost and infrastructure, and ethical considerations such as patient privacy and data security in implementing metaverse technology in medical education. It also omits the perspectives of educators, healthcare providers, and patients on integrating this technology.\\
\textbf{Results:}\\ The paper focuses on methods for revolutionizing medical education using metaverse. It discusses the potential of metaverse in providing interactive and immersive experiences in healthcare.
\section{The Metaverse in Medical Education and Clinical Practice\cite{juan2023metaverse}}
\textbf{Methodology:}\\Use of immersive virtual and augmented environments.
Utilization of different models of stereoscopic vision glasses\\
\textbf{Limitations:}\\No specific limitations mentioned in the abstract. Further details on limitations not provided in the text.\\
\textbf{Results:}\\The paper presents immersive virtual and augmented environments for medical training. These environments use stereoscopic vision glasses to create a metaverse for enhanced medical training.
\section{Virtual reality and the transformation of medical education\cite{pottle2019virtual}}
\textbf{Methodology:}\\The paper explores simulation's role in clinical training, noting its efficacy but resource-intensive nature. It highlights the shift to virtual reality (VR) for medical education, offering cost-effective, repeatable training. Evidence supports VR's effectiveness across industries, including healthcare. Integrating VR into curricula is vital for medical education's future, enabling shared clinical experiences globally. VR's implementation is expanding as an educational tool, transforming future clinicians' training by providing quality, geographically independent education.\\
\textbf{Limitations:}\\
The paper may generalize the effectiveness of VR in medical education without addressing specific limitations or challenges. It lacks an in-depth analysis of potential barriers to widespread adoption and does not cover all possible counterarguments, presenting a one-sided view. The paper also fails to identify areas for further research to address VR limitations and overlooks technological constraints crucial for real-world implementation.\\
\textbf{Results:}\\Digital transformation has a great impact on medical education.
Inclusion of AI and VR benefits medical students.
\section{A Virtual Reality for the Digital Surgeon\cite{velazquez2021virtual}}
\textbf{Methodology:}\\The paper conducted a literature review on VR technology, its healthcare applications, surgical education, and support tools. It analyzed VR's capabilities, advancements, and benefits for enhancing surgical training and preoperative planning. The evaluation highlighted VR's potential to optimize patient-specific data, improve surgical outcomes, and revolutionize healthcare.\\
\textbf{Limitations:}\\The paper's scope is limited, potentially restricting the generalizability of its findings on VR in healthcare. It may overlook certain challenges or barriers to implementing VR, indicating a research gap that leaves room for further investigation. Additionally, reliance on a limited dataset or specific sources could impact the depth and reliability of the findings, affecting the robustness of the conclusions about VR's impact on surgical education and clinical practice.\\
\textbf{Results:}\\ VR has the potential to improve surgical education and skills.
VR can optimize preoperative planning and intraoperative support in clinical practice.
\section{Transforming medical education and training with VR using M.A.G.E.S.\cite{ProceedingsArticle}}
\textbf{Methodology:}\\The paper introduces a novel VR software system for healthcare training, focusing on Psychomotor Virtual Reality Surgical Training. It provides a realistic, fail-safe environment for surgeons to enhance skills through gamification and advanced interactability. The system supports multiple surgeons and assistants for cooperative operations, using a custom CGA GPU interpolation engine to minimize data transfer.\\
\textbf{Limitations:}\\The paper focuses on orthopedic surgeries, limiting generalizability. Cost-effectiveness for widespread adoption is not discussed. The VR surgical training solution's validation and integrated educational curriculum effectiveness are not detailed. Long-term skill retention and user experience feedback are also omitted.\\
\textbf{Results:}\\The paper focuses on methods for revolutionizing medical education using metaverse. It discusses the potential of metaverse in providing interactive and immersive experiences in healthcare.
\section{Virtual Reality in Medicine\cite{JournalArticle}\cite{JournalArticle11}}
\textbf{Methodology}\\Multimodal interactions between user and virtual environment.
Technical requirements and design principles of input devices, displays, and rendering techniques.\\
\textbf{Limitations:}\\Physiological constraints. Technical requirements and design principles of multimodal input devices, displays, and rendering techniques.\\
\textbf{Results:}\\ Examples of virtual reality applications in surgical training, intra-operative augmentation, and rehabilitation. Provides technical requirements and design principles for virtual reality in medicine.
\section{A Virtual Environment for Training and Assessment of Surgical Teams\cite{papagiannakis2018virtual}}
\textbf{Methodology:}\\The paper explores Collaborative Virtual Environments (CVEs) for enhancing remote interactions in surgical team training and assessment. CVEs simulate medical procedures, allowing remote users to train together, improving education and teamwork skills. VR in education reduces costs and the need for live subjects, offering interactive teaching. The proposed CVE architecture supports team training and assessment in surgical simulations, aiming to develop team skills for effective surgical teamwork.\\
\textbf{Limitations:}\\Cost reduction for training is a limitation.
Use of guinea pigs and anatomical specimens is reduced.\\
\textbf{Results:}\\Proposed architecture for training and assessing team skills during surgery. Use of statistical models to monitor and assess team performance.
\section{Virtual reality technology and its application in modern medicine
\cite{JournalArticle}}
\textbf{Methodology:}\\The paper reviews VR technology and its medical applications, including virtual human simulations, assisted diagnosis, surgery, and telemedicine. It explores VR's use in surgical training, patient education, and remote consultations, illustrating practical implications through case studies. The paper compares VR with traditional methods, highlighting advantages and limitations, and includes expert interviews on current trends, challenges, and future prospects in healthcare.\\
\textbf{Limitations:}\\Limited access to VR technology in medical settings. Challenges in integrating VR into existing medical practices.\\
\textbf{Results:}\\ VR applied in virtual human, assisted diagnosis, surgery simulation. VR used in virtual telemedicine for medical purposes.
\section{Role of virtual reality for healthcare education
\cite{BookChapter}}
\textbf{Methodology:}\\The research paper explores VR technology in healthcare education, highlighting its role in storing patient data as 3D points for virtual access to anatomy and clinical outcomes. It enables medical students to analyze anatomy from various angles and practice contactless procedures, enhancing clinical skills. Additionally, VR provides flexible training for practitioners, allowing them to visualize the human body and rehearse complex operations, thus improving medical education quality and standards.\\
\textbf{Limitations:}\\
Few are using VR to evaluate medical students' success. VR technology not widely used for medical training evaluation.\\
\textbf{Results:}\\ VR improves learning and training of medical practitioners.
VR enhances comprehension of anatomy and clinical outcomes.
\section{Next-Gen Mulsemedia: Virtual Reality Haptic Simulator’s Impact on Medical Practitioner for Higher Education Institutions\cite{journalarticle4}\cite{JournalArticle5}\cite{JournalArticle8}}
\textbf{Methodology:}\\ The study used the core motivation hypothesis to boost motivation in the classroom. The study used the attention, relevance, confidence, and satisfaction (ARCS) model to analyze the impact of virtual reality on student motivation and content update implementation.\\
\textbf{Limitations:}\\ Lack of research on consequences of virtual reality
Early stage of virtual reality technology research.\\
\textbf{Results:}\\ Virtual reality simulators improve student motivation and learning. VR has the potential to transform medical education.

\section{Design and implementation of a 3D digestive teaching system based on virtual reality technology in modern medical education\cite{Journal1}}
\textbf{Methodology:}\\
DX technologies: VR, AR, MR, XR, 3D images, holograms, AI. Utilization of HMDs, wearable sensors, 5G, and Wi-Fi.\\
\textbf{Limitations:}\\ Lack of systematic methodology, affecting evidence level
Heterogeneity in XR techniques studies, limiting systematic reviews.\\
\textbf{Results:}\\ 

\section{A Virtual Operating Room for Context-Relevant Training \cite{JournalArticle2}}
\textbf{Methodology:}\\
Virtual agents with unique personalities and knowledge structures defined. Immersive Virtual Operating Room (VOR) simulating surgical procedures described \\
\textbf{Limitation:}\\
Current medical simulators lack addressing errors in healthcare system. Existing simulators focus on procedural skills, not team dynamics.\\
\textbf{Results:}\\ Pilot session with surgical resident unfamiliar with VOR showed challenges. VOR allows team-based surgical training with virtual expert agents.

\section{Virtual reality surgical training and assessment system\cite{JournalArticle3}\cite{JournalArticle7}}
\textbf{Methodology:}\\ Soft-tissue model based on volumetric mass-spring system
Texture mapping for realistic organ representation and space perception.\\
\textbf{Limitation:}\\ Limits of realism in surgical simulation Challenges in obtaining reliable measures of surgical skills.\\
\textbf{Results:}\\ VR surgical system based on C-source code and OpenGL.
System handles accurate interactions between soft-tissue and surgical instruments