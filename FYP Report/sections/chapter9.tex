\chapter{Iteration 4}
\label{ch:iter4}
The Meta-Med VR project's fourth iteration aims to improve user experience by integrating Keyboard and Meta Quest 2 functionalities. Keyboard provides input methods for medical notes and virtual objects, while Meta Quest 2 offers advanced hand tracking and gesture recognition for natural interaction. This enhances immersion, realism, and educational value, making it a significant step forward in healthcare professionals' training tools.
\section{Introduction}
\textbf{Overview of $MetaMed$ VR Project:}\\
Meta-Med VR is a revolutionary approach that aims to enhance medical education by providing immersive virtual reality experiences for training and skill development.\\
\textbf{Integration Focus:}\\
This document focuses on the integration of Keyboard and Meta Quest 2 to enhance object interaction and movement within the $Meta Med$ VR environment, aiming to improve user engagement and learning outcomes.
\section{Integration Process with Keyboard}
We evaluated several input devices based on factors such as compatibility, affordability, and ease of use. After careful consideration, we chose Keyboard for its versatility and familiarity to users.\\
\textbf{Compatibility Assessment:}\\
The $MetaMed$ VR software underwent rigorous compatibility tests to ensure seamless integration, verifying hardware specifications and assessing software support for keyboard inputs.\\
\textbf{Software Configuration:}\\
Custom software configurations were created to enable recognition and response to keyboard inputs in the $MetaMed$ VR environment, mapping keyboard keys to specific actions or functions.\\
\textbf{Testing and Calibration:}\\
The integration underwent rigorous testing to ensure smooth object interaction and movement, with calibration procedures implemented to optimize performance and reduce input latency.\\
\textbf{Enhancements and Features:}\\ 
Users can manipulate virtual objects using keyboard inputs, performing actions like grabbing, rotating, and moving objects within the VR space.\\
\textbf{Movement Controls:}\\
The keyboard integration offers users intuitive movement controls, including locomotion and teleportation options, facilitating seamless navigation through virtual environments.\\
\textbf{User Experience Improvements:}\\
The integration improves immersion and realism in the $MetaMed$ VR environment, enhancing engagement and effectiveness in training scenarios.
\section{Integration Process with Meta Quest 2}
\textbf{Selection of Meta Quest 2:}\\
We selected Meta Quest 2 due to its advanced features and user-friendly interface after evaluating various input devices for compatibility, affordability, and tracking capabilities.\\
\textbf{Compatibility Assessment:}\\
The integration of $MetaMed$ VR software was ensured through rigorous compatibility tests, verifying hardware specifications and assessing software support for Meta Quest 2 inputs.\\
\textbf{Software Configuration:}\\
Custom software configurations were created to enable recognition and response to Meta Quest 2 inputs within the $MetaMed$ VR environment, including integrating Meta Quest 2.
\section{Testing and Calibration}
\textbf{Testing Procedures:}\\
The integration underwent rigorous testing to ensure smooth object interaction and movement, with calibration procedures implemented to optimize tracking accuracy and minimize latency.
\section{Enhancements and Features}
\textbf{Object Interaction:}\\
Meta Quest 2 enables users to interact with virtual objects through gestures like grabbing, rotating, and moving within the VR space.\\
\textbf{Movement Controls:}\\
Meta Quest 2's integration offers users intuitive movement controls, including locomotion and teleportation options, facilitating seamless navigation through virtual environments.\\
\textbf{User Experience Improvements:}\\
The integration improves immersion and realism in the $MetaMed$ VR environment, enhancing engagement and effectiveness in training scenarios.
\section{Challenges and Solutions}
\textbf{Technical Challenges:}\\
The integration process faced technical issues like input latency and mapping conflicts.\\
\textbf{Solutions Implemented:}\\
The challenges were resolved through software updates, configuration adjustments, and calibration procedures to guarantee a seamless user experience.
\section{Future Directions}
\textbf{Potential Enhancements:}\\
Future improvements include improving input mappings, enhancing keyboard shortcut support, improving gesture recognition algorithms, and introducing advanced hand tracking features.
\section{Conclusion}
\textbf{Summary of Achievements:}\\
The integration of Keyboard and Meta Quest 2 with $MetaMed$ VR is a significant advancement in medical education through immersive virtual reality experiences.\\
\textbf{Impact and Significance:}\\
$MetaMed$ VR enhances healthcare professional training and patient care outcomes by improving object interaction and movement controls.\\
