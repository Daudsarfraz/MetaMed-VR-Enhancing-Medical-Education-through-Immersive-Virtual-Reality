\chapter{Software Requirements Specification}
\label{ch:srs}
This chapter will contain the functional and non-functional requirements of the project.
\section{List of functions}
List of key features for the "Improving Medical Education through Immersive Virtual Reality (MetaMed)" project:\\
\textbf{Realistic medical simulations:}\\ Create highly realistic medical scenarios and procedures that faithfully mimic real medical experiences, including various medical specialties.\\
\textbf{Haptic feedback integration:}\\ Integrate tactile feedback devices to provide tactile sensations, increasing the realism of medical training and allowing students to develop their tactile skills.\\
\textbf{Immersive 3D Environments:}\\ Create immersive, 3D virtual environments that accurately represent medical settings, including operating rooms, medical instruments, and anatomical structures.\\
\textbf{Interactivity:}\\Enables hands-on interaction with virtual medical tools, instruments, and equipment, allowing students to perform medical tasks and procedures.\\
\textbf{Remote Access:}\\ Enable medical students to access a metaverse-based medical training platform from remote locations, facilitating flexible learning.
\section{Test plan (test level, test techniques)}
\text{Test plan for our project will be as:}
\begin{itemize}
    \item Movement in VR 
    \item Interaction with Objects
    \item Treatement to Patient
\end{itemize}
\section{Use Case Diagram}
Here is the Use Case Diagram of the Project.
\begin{figure}[h]
    \centering
    \includegraphics[width=0.5\linewidth]{Images/Use Case.drawio.png}
    \caption{Use Case Diagram}
    \label{fig:system-diagram}
\end{figure}

\section{Software Development Plan}
\text{Software Development Plan for our project will be as:}
\begin{itemize}
    \item Project Overview 
    \item Requirements Gathering and Analysis
    \item System Design
    \item Development
    \item Testing
    \item Documentation
    \item Management
\end{itemize}

\section{Sequence Daigram/System Diagram}
It is the Sequence Daigram/System Diagram of the Project.
\begin{figure}[h]
    \centering
    \includegraphics[width=1\linewidth, height=0.65\linewidth]{Images/system.png}
    \caption{System Diagram}
\end{figure}
\newpage
\section{Activity Daigram}
The activity diagram aims to illustrate the entire process of project working and demonstration.
\begin{figure}[h]
    \centering
    \includegraphics[width=0.25\linewidth]{Images/Activity.drawio.png}
    \caption{Activity Diagram}
\end{figure}

\section{Tools and Technologies used}
\subsection{C-Sharp:}
C-Sharp is a programming language developed by Microsoft that runs on the .NET Framework. C-Sharp is used to develop web apps, desktop apps, mobile apps, games and much more.
\begin{figure}[h]
	\centering
	\includegraphics[width=0.2\linewidth, height=0.2\linewidth]{Images/CSharp.png}
	\caption{C-Sharp\cite{csharp}}
\end{figure}
\subsection{Blender}
Blender is a free, open-source 3D graphics software for creating animated films, visual effects, art, 3D models, motion graphics, interactive applications, and virtual reality. It features 3D modeling, UV mapping, texturing, digital drawing, rigging, simulations, sculpting, animation, rendering, video editing, and compositing.
\begin{figure}[h]
	\centering
	\includegraphics[width=0.3\linewidth, height=0.3\linewidth]{Images/blender.png}
	\caption{blender\cite{blender}}
\end{figure}
\subsection{Git}
Git is a distributed version control system that tracks versions of files. It is used to control source code and collaboratively developing software.
\begin{figure}[h]
	\centering
	\includegraphics[width=0.2\linewidth, height=0.2\linewidth]{Images/git.png}
	\caption{git\cite{git-logo}}
\end{figure}
\subsection{GitHub}
GitHub is a developer platform that allows developers to create, store, manage and share their code. We used Git software, providing the distributed version control of Git plus access control, task management, continuous integration.
\begin{figure}[h]
	\centering
	\includegraphics[width=0.2\linewidth, height=0.2\linewidth]{Images/GitHub.png}
	\caption{GitHub\cite{github-logo}}
\end{figure}
\subsection{Meta}
"Meta" in the Metaverse represents an advanced, interconnected virtual world combining VR. It enhances user experiences by creating seamless, immersive digital environments.
\begin{figure}[h]
	\centering
	\includegraphics[width=0.2\linewidth, height=0.2\linewidth]{Images/meta.png}
	\caption{Meta\cite{meta}}
\end{figure}
\subsection{Unity}
We used Unity to create three-dimensional (3D) and two-dimensional (2D) games, as well as interactive simulations. The engine has been adopted by industries outside video gaming.
\begin{figure}[h]
	\centering
	\includegraphics[width=0.2\linewidth,height=0.2\linewidth]{Images/unity.png}
	\caption{Unity\cite{Unity}}
\end{figure}
\newpage
\subsection{VRTK (Virtual Reality Tool Kit)}
We use it for locomotion within virtual space, interactions like touching, grabbing, and using objects, interacting with Unity3D UI elements through pointers or touch, body physics within virtuatextbfl space, and 2D and 3D controls such as buttons, levers, doors, and drawers.
\begin{figure}[h]
	\centering
	\includegraphics[width=0.3\linewidth, height=0.2\linewidth]{Images/vrtk.png}
	\caption{VRTK (Virtual Reality Tool Kit)\cite{vrtk}}
\end{figure}
\subsection{Meta Quest 2}
We chose the Meta Quest 2 for Project MetaMed VR to enhance medical education through immersive virtual reality due to its advanced features like hand tracking, gesture recognition, and immersive VR experiences, which improve user interaction and realism in medical simulations.
\begin{figure}[h]
	\centering
	\includegraphics[width=0.4\linewidth,height=0.3\linewidth]{Images/metaquest2.png}
	\caption{Meta Quest 2\cite{metaquest}}
\end{figure}
\newpage
\subsection{Microsoft Visual C++}
Microsoft Visual C++ is used with Unity 3D primarily for its ability to create highly optimized code that improves performance and memory management in complex applications. This integration ensures that Unity games and applications run smoothly and efficiently, even with demanding graphics and processing requirements.
\begin{figure}[h]
	\centering
	\includegraphics[width=0.2\linewidth,height=0.2\linewidth]{Images/VS Code.png}
	\caption{Microsoft Visual C++\cite{visual-studio-icon}}
\end{figure}