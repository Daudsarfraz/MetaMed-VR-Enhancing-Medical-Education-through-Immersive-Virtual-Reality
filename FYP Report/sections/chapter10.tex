\chapter{Implementation Details}
\label{ch:implementation}
{Here are some implementation details of project. What we can do in project.}
\section{Introduction}
This essay explores the use of Meta Quest 2 and the various Libraries to create a comprehensive virtual training environment for medical practitioners, detailing the process from user authentication to performing procedures in a virtual hospital setting, showcasing Virtual Reality's potential in medical education.
\section{User Authentication and Environment Access}
The training uses Meta Quest 2, a high-resolution virtual environment, with keyboard navigation. Authentication involves signing up through the meta-verse or Meta-Mask, ensuring security. Correct credentials grant access, while incorrect ones prompt a new sign-up, ensuring integrity and integrity.
\section{Navigation to the Virtual Hospital Environment}
Upon successful authentication, users are transported to a meticulously crafted virtual hospital environment, allowing navigation through various areas such as wards, operation theaters, laboratories, and administrative sections, enhancing the training experience.
\section{Scenario Selection and Patient Interaction}
Users visit the reception area to select an interactive training scenario from various patient cases, covering various medical conditions. They then practice diagnostic and treatment skills in a controlled, risk-free environment by observing detailed patient histories, symptoms, and physical examinations in the designated patient.
\section{Laboratory Interaction and Tool Acquisition}
Users choose a patient scenario and visit a virtual laboratory to gather tools and medications. The VR environment simulates a fully-equipped lab, allowing users to practice selecting and handling medical instruments. Some open-source frameworks enhance realism and training effectiveness.
\section{Performing Medical Procedures}
Users use VR to perform medical procedures, providing step-by-step guidance and real-time feedback. The immersive nature allows users to experience procedural nuances, such as injections, wound suturing, and diagnostic equipment usage.
\section{Exit from the Virtual Environment}
The user exits the virtual environment after completing the training scenario, allowing easy transition back to the real world, and the training sessions can be repeated for skill development and refinement.

